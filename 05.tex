\chapter{考察}
\label{chap:consideration}

\section{本研究についての考察}
現在、多くのデータが溢れビッグデータとして価値が見直されている。特にソーシャルメディアの中でのユーザーデータの価値はどんどんと上げっている。ユーザーがセンシングすることの出来るデータも私は今後重要になってくるだろう。現状、位置情報を写真やSNSの発信地として付加する程度にしか活用されていない。さらに言えばこれらの情報をユーザー自身が使うことは稀だ。SNSを運営している企業やアプリケーション側が整理や管理のために使っているケースがほとんどだ。本人にとっても価値のあるデータを自身が使えるようにするだけで便利になることは間違いないだろう。さらに言えば、センシングデータは個人だけでなく共有出来るところにも価値がある。ひとりのユーザーがセンシングした東京の気温のデータを他の全てのユーザーが共有することで集合知的に情報が集まる。そのようにして集めた全てのデータをユーザーが自由に使えるようにすることは新しい価値の創造の一助となるだろう。

本研究はLindaを利用したデータの利用法を提案したが、そもそもクラウド上にあらゆるデータをあげておくことに抵抗のあるユーザーも多いだろう。そのためにもデータの可視化は必要である。自動で写真に位置情報が付加されるなど自分がセンシングしているにも関わらずその事実を知らないことも多い。自動というキーワードはユビキタスコンピューティングのようで便利ではあるのだがその反面意図せずに実行してしまうことも多々ある。今回、センシングしているデータを可視化することでどのようなデータをセンシングし他人と共有しているのか、また自分が使うことが出来るのかということを把握することで安心して利用することが出来る環境をつくることができた。

\section{議論}
\subsection{ビジュアライジング手法について}
ビジュアライジング手法についてどのような表現が一番適切かという議論があった。Pure data\footnote{http://puredata.info/}(以下Pd)というオープンソースのソフトがある。Pdは主に音楽制作のMaxなどに使われておりArduino\footnote{http://arduino.cc/}などをビジュアルプログラミングをする際などにも利用されることが多い。
今回、Pdを使ってプロトタイプを作成することも議論になった。Javascriptで利用できるWebPd\footnote{https://github.com/sebpiq/WebPd}の利用も考えたが、結局コネクションの貼り方等で独自に指定する必要があったため実装の労力と自由度を考え、今回は一から自分で実装することにした。必要なオブジェクトとコネクションの概念だけで良かったのでよりシンプルな実装が出来た。

\subsection{センシングデータの利用法について}
本研究を始めるにあたりセンシングデータをどのように利用するか、が議論としてあった。センシングが社会に浸透し始め、データはとれるようになった。しかしそのデータをどのように利用するかが考えられておらず、実際にもセンシングデータを有効に利用したアイデアやアプリケーションなどは実装されていない。本研究は実際にどのように利用するのかという問題を解決できた訳ではない。しかしユーザビリティを向上させ直感的に操作できるようにすることで、センシングデータを一個人が利用するハードルを下げることができたのではないだろうか。これから、これらのデータがどのように利用されていくかが本当の課題になってくるだろう。

