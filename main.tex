% 独自のコマンド

% ■ アブストラクト
%  \begin{jabstract} 〜 \end{jabstract}  :日本語のアブストラクト
%  \begin{eabstract} 〜 \end{eabstract}  :英語のアブストラクト

% ■ 謝辞
%  \begin{acknowledgment} 〜 \end{acknowledgment}

% ■ 文献リスト
%  \begin{bib}[100] 〜 \end{bib}


\newif\ifjapanese

\japanesetrue  % 論文全体を日本語で書く(英語で書くならコメントアウト)

\ifjapanese
  %\documentclass[a4j,twoside,openright,11pt]{jreport} % 両面印刷の場合。余白を綴じ側に作って右起こし。
  \documentclass[a4j,11pt]{jreport}                  % 片面印刷の場合。
  \renewcommand{\bibname}{参考文献}
  \newcommand{\acknowledgmentname}{謝辞}
\else
  \documentclass[a4paper,11pt]{report}
  \newcommand{\acknowledgmentname}{Acknowledgment}
\fi
\usepackage{thesis}
\usepackage{ascmac}
\usepackage{graphicx}
\usepackage{multirow}
\usepackage{url}
\bibliographystyle{jplain}

%\bindermode  % バインダー用余白設定

% 日本語情報(必要なら)
\jclass  {卒業論文}                             % 論文種別
\jtitle    {センシングデータを可視化する\\フレームワークの作成とその利用法の提案}    % タイトル。改行する場合は\\を入れる
\juniv    {慶應義塾大学}                  % 大学名
\jfaculty  {環境情報学部環境情報学科}               % 学部、学科
\jauthor  {永倉 啓太}                       % 著者
\jhyear  {25}                                   % 平成○年度
\jsyear  {2013}                                 % 西暦○年度
\jkeyword  {センシングデータ、ビジュアルプログラミング、Linda、データ可視化}     % 論文のキーワード
\jproject{増井俊之研究会} %プロジェクト名
\jdate{2014年1月}

% 英語情報(必要なら)


\begin{document}

\ifjapanese
  \jmaketitle    % 表紙(日本語)
\else
  \emaketitle    % 表紙(英語)
\fi

% ■ アブストラクトの出力 ■
%	◆書式:
%		begin{jabstract}〜end{jabstract}	:日本語のアブストラクト
%		begin{eabstract}〜end{eabstract}	:英語のアブストラクト
%		※ 不要ならばコマンドごと消せば出力されない。



% 日本語のアブストラクト
\begin{jabstract}
多くのデータがセンサーでとれる時代になったが、未だに個人で自由にセンシングデータを使える環境が整っていない。

そこで本研究ではセンシングデータを可視化し、データを簡単に取り出して利用できるフレームワークを作成した。本研究ではLindaという分散型プログラミングができるフレーワークを拡張した形になっている。Lindaは様々な場所で様々なユーザーが取得したセンシングデータをWeb上に流し、誰でも共有して利用できるようにしている。それらのセンシングデータをより簡単に利用できるようにすることで多くのプロダクトが産まれる可能性が広がるだろう。

また、本研究ではセンシングデータを物に結びつけてプッシュAPI化する実装となっている。センシングデータを物と結びつけることでよりモジュール化し、より直感的に再利用できるメリットがある。

\end{jabstract}



% 英語のアブストラクト
  % アブストラクト。要独自コマンド、include先参照のこと

\tableofcontents  % 目次
\listoffigures    % 表目次
\listoftables    % 図目次

\pagenumbering{arabic}

\include{01}  % 本文1
\include{02}  % 本文2
\include{03}  % 本文3
\include{04}  % 本文4
\chapter{考察}
\label{chap:consideration}

\section{本研究についての考察}
現在、多くのデータが溢れビッグデータとして価値が見直されている。特にソーシャルメディアの中でのユーザーデータの価値はどんどんと上げっている。ユーザーがセンシングすることの出来るデータも私は今後重要になってくるだろう。現状、位置情報を写真やSNSの発信地として付加する程度にしか活用されていない。さらに言えばこれらの情報をユーザー自身が使うことは稀だ。SNSを運営している企業やアプリケーション側が整理や管理のために使っているケースがほとんどだ。本人にとっても価値のあるデータを自身が使えるようにするだけで便利になることは間違いないだろう。さらに言えば、センシングデータは個人だけでなく共有出来るところにも価値がある。ひとりのユーザーがセンシングした東京の気温のデータを他の全てのユーザーが共有することで集合知的に情報が集まる。そのようにして集めた全てのデータをユーザーが自由に使えるようにすることは新しい価値の創造の一助となるだろう。

本研究はLindaを利用したデータの利用法を提案したが、そもそもクラウド上にあらゆるデータをあげておくことに抵抗のあるユーザーも多いだろう。そのためにもデータの可視化は必要である。自動で写真に位置情報が付加されるなど自分がセンシングしているにも関わらずその事実を知らないことも多い。自動というキーワードはユビキタスコンピューティングのようで便利ではあるのだがその反面意図せずに実行してしまうことも多々ある。今回、センシングしているデータを可視化することでどのようなデータをセンシングし他人と共有しているのか、また自分が使うことが出来るのかということを把握することで安心して利用することが出来る環境をつくることができた。

\section{議論}
\subsection{ビジュアライジング手法について}
ビジュアライジング手法についてどのような表現が一番適切かという議論があった。Pure data\footnote{http://puredata.info/}(以下Pd)というオープンソースのソフトがある。Pdは主に音楽制作のMaxなどに使われておりArduino\footnote{http://arduino.cc/}などをビジュアルプログラミングをする際などにも利用されることが多い。
今回、Pdを使ってプロトタイプを作成することも議論になった。Javascriptで利用できるWebPd\footnote{https://github.com/sebpiq/WebPd}の利用も考えたが、結局コネクションの貼り方等で独自に指定する必要があったため実装の労力と自由度を考え、今回は一から自分で実装することにした。必要なオブジェクトとコネクションの概念だけで良かったのでよりシンプルな実装が出来た。

\subsection{センシングデータの利用法について}
本研究を始めるにあたりセンシングデータをどのように利用するか、が議論としてあった。センシングが社会に浸透し始め、データはとれるようになった。しかしそのデータをどのように利用するかが考えられておらず、実際にもセンシングデータを有効に利用したアイデアやアプリケーションなどは実装されていない。本研究は実際にどのように利用するのかという問題を解決できた訳ではない。しかしユーザビリティを向上させ直感的に操作できるようにすることで、センシングデータを一個人が利用するハードルを下げることができたのではないだろうか。これから、これらのデータがどのように利用されていくかが本当の課題になってくるだろう。

  % 本文5
\include{06}  % 本文6

\include{90_acknowledgment}  % 謝辞。要独自コマンド、include先参照のこと
\include{91_bibliography}  % 参考文献。要独自コマンド、include先参照のこと
\appendix
%\include{92_appendix}    % 付録

\end{document}
