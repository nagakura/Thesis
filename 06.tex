\chapter{結論}
\label{chap:conclusion}
本研究ではセンシングデータを利用するためのフレームワークの作成を目指した。プロトタイプを作り、実際にビジュアルプログラミングをさせることでセンサー利用を促すことができたはずだ。少なくとも、データの可視化はユーザーのハードルを下げられただろう。今後はスマートフォンからの利用を考え、もっと手軽で軽量なフレームワークを目指していきたい。また、実際に利用する際にどうしてもセンサーデータをプログラミングするという手間が出来てしまう。いつもと違うセンサーデータによって条件を指定する時など、人力での指定は難しい。センサーデータを学習させることで異常値を検出させるなどの技術を盛り込むとより快適なセンシング社会を実現することが出来るだろう。

\cite{source01}の研究にもあるようにセンサー利用は生活に直結する重要な基礎技術である。本研究が今後のセンシング社会に少しでも貢献できたら幸いである。

