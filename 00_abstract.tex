% ■ アブストラクトの出力 ■
%	◆書式:
%		begin{jabstract}〜end{jabstract}	:日本語のアブストラクト
%		begin{eabstract}〜end{eabstract}	:英語のアブストラクト
%		※ 不要ならばコマンドごと消せば出力されない。



% 日本語のアブストラクト
\begin{jabstract}
多くのデータがセンサーでとれる時代になったが、未だに個人で自由にセンシングデータを使える環境が整っていない。

そこで本研究ではセンシングデータを可視化し、データを簡単に取り出して利用できるフレームワークを作成した。本研究ではLindaという分散型プログラミングができるフレーワークを拡張した形になっている。Lindaは様々な場所で様々なユーザーが取得したセンシングデータをWeb上に流し、誰でも共有して利用できるようにしている。それらのセンシングデータをより簡単に利用できるようにすることで多くのプロダクトが産まれる可能性が広がるだろう。

また、本研究ではセンシングデータを物に結びつけてプッシュAPI化する実装となっている。センシングデータを物と結びつけることでよりモジュール化し、より直感的に再利用できるメリットがある。

\end{jabstract}



% 英語のアブストラクト
