\chapter{序論}
\label{chap:introduction}
\section{背景}
近年、データのセンシングが簡単になり多くのセンシングデータがビッグデータとしてあふれている。iPhoneやAndroidなどのスマートフォン端末でも位置情報や加速度、傾きなどのデータを常にセンシングすることが出来る。それに伴い、これらのセンシングデータを利用したアプリケーションが出されたり、研究目的として利用されたりするようになった。

しかしながら、これらのセンサーデータを簡単に利用できるようなフレームワークが現状存在しておらずデータをとっても活用せずに終わりといったケースが多かった。使い捨てのデータになっているこれらのデータを活用しない手はないだろう。

また、センサーデータを利用するアプリケーションは多数存在するが、個人がそれらのデータを使っているシーンは少ない。これは個人利用のハードルが高いためだ。多くの場合、プログラミング手段を用いてセンシングデータにアクセスしている。

まずはセンシングデータを可視化することでエンドユーザーに知らせることが必要だ。次点としてセンシングデータを簡単に利用できるように支援する必要がある。

個人が簡単にセンシングデータを利用できるようにするだけでより便利な世の中になるのではないだろうか。

\section{目的}
世の中に氾濫しているセンシングによるビッグデータをより効果的に利用するためのフレームワークを作成する。

そのためにまずはセンシングデータを理解できる数値、文字に変換することが必要になってくる。可視化できていなければどんなデータが利用できるかわからないためだ。

センシングデータを利用するためのフレームワークを作成する。これまでプログラミングができないとセンシングデータを利用できないケースが多くあり課題だったため、プログラミングをせずに使える直感的で使いやすいインタフェースを目指した。GUIでの操作は[1]の研究にもあるようにプログラミングせずに操作する有用な手法であると考えた。

また、他人が作ったセンシングデータをさらに再利用して使えるような仕組み作りをし、汎用的に多くの場面で使えるように想定した。

\section{本文書の構成}
第\ref{chap:introduction}章では本研究の概要を書いた。

第\ref{chap:contents}章では研究内容を説明する。第\ref{chap:prototype}章ではプロトタイプの実装方法を解説する。第\ref{chap:consideration}章では考察を書く。最後に第\ref{chap:conclusion}章にて結論を書き本論文をしめることとする。添付として参考文献を追記する。
